\documentclass[varwidth=\maxdimen,border=10]{standalone}
\begin{document}
The group $G$ is isomorphic to the projective special linear group PSL(2,8).\\
Ordinary character table of $G$\ $\cong$\ PSL(2,8):\\
\begin{center}
\begin{tabular}{@{}l@{}l@{}l@{}}
\hline
\(\begin{array}{|l|ccccccccc|}
  & 1a & 2a & 3a & 7a & 7b & 7c & 9a & 9b & 9c\\ \hline
\chi_{1} & 1 & 1 & 1 & 1 & 1 & 1 & 1 & 1 & 1\\
\chi_{2} & 7 & -1 & -2 & 0 & 0 & 0 & 1 & 1 & 1\\
\chi_{3} & 7 & -1 & 1 & 0 & 0 & 0 & E(9)^{2}+E(9)^{4}+E(9)^{5}+E(9)^{7} & -E(9)^{2}-E(9)^{7} & -E(9)^{4}-E(9)^{5}\\
\chi_{4} & 7 & -1 & 1 & 0 & 0 & 0 & -E(9)^{4}-E(9)^{5} & E(9)^{2}+E(9)^{4}+E(9)^{5}+E(9)^{7} & -E(9)^{2}-E(9)^{7}\\
\chi_{5} & 7 & -1 & 1 & 0 & 0 & 0 & -E(9)^{2}-E(9)^{7} & -E(9)^{4}-E(9)^{5} & E(9)^{2}+E(9)^{4}+E(9)^{5}+E(9)^{7}\\
\chi_{6} & 8 & 0 & -1 & 1 & 1 & 1 & -1 & -1 & -1\\
\chi_{7} & 9 & 1 & 0 & E(7)+E(7)^{6} & E(7)^{2}+E(7)^{5} & E(7)^{3}+E(7)^{4} & 0 & 0 & 0\\
\chi_{8} & 9 & 1 & 0 & E(7)^{3}+E(7)^{4} & E(7)+E(7)^{6} & E(7)^{2}+E(7)^{5} & 0 & 0 & 0\\
\chi_{9} & 9 & 1 & 0 & E(7)^{2}+E(7)^{5} & E(7)^{3}+E(7)^{4} & E(7)+E(7)^{6} & 0 & 0 & 0\\
\hline
\end{array}\)\\
\end{tabular}
\end{center}
\begin{tabular}{@{}l@{}l@{}l@{}l@{}l@{}l@{}l@{}l@{}l@{}l@{}}
Trivial source character table of $G$\ $\cong$\ PSL(2,8) at\ $p=3$:\\
\(\begin{array}{|l|ccccc|cc|cc|}
\hline
\textup{Normalisers}\ N_i & \multicolumn{5}{c|}{N_{1}} & \multicolumn{2}{c|}{N_{2}} & \multicolumn{2}{c|}{N_{3}}\\ \hline
p\textup{-subgroups\ of\ } G\ \textup{up\ to\ conjugacy\ in\ } G & \multicolumn{5}{c|}{P_{1}} & \multicolumn{2}{c|}{P_{2}} & \multicolumn{2}{c|}{P_{3}}\\ \hline
\textup{Representatives}\ n_j\ \in\ N_i & 1a & 7a & 7c & 7b & 2a & 1a & 2a & 1a & 2a\\ \hline
{1}\cdot \chi_{1}+{0}\cdot \chi_{2}+{0}\cdot \chi_{3}+{0}\cdot \chi_{4}+{0}\cdot \chi_{5}+{1}\cdot \chi_{6}+{0}\cdot \chi_{7}+{0}\cdot \chi_{8}+{0}\cdot \chi_{9} & 9 & 2 & 2 & 2 & 1 & 0 & 0 & 0 & 0\\
{0}\cdot \chi_{1}+{1}\cdot \chi_{2}+{1}\cdot \chi_{3}+{1}\cdot \chi_{4}+{1}\cdot \chi_{5}+{1}\cdot \chi_{6}+{0}\cdot \chi_{7}+{0}\cdot \chi_{8}+{0}\cdot \chi_{9} & 36 & 1 & 1 & 1 & -4 & 0 & 0 & 0 & 0\\
{0}\cdot \chi_{1}+{0}\cdot \chi_{2}+{0}\cdot \chi_{3}+{0}\cdot \chi_{4}+{0}\cdot \chi_{5}+{0}\cdot \chi_{6}+{0}\cdot \chi_{7}+{0}\cdot \chi_{8}+{1}\cdot \chi_{9} & 9 & E(7)^{2}+E(7)^{5} & E(7)+E(7)^{6} & E(7)^{3}+E(7)^{4} & 1 & 0 & 0 & 0 & 0\\
{0}\cdot \chi_{1}+{0}\cdot \chi_{2}+{0}\cdot \chi_{3}+{0}\cdot \chi_{4}+{0}\cdot \chi_{5}+{0}\cdot \chi_{6}+{0}\cdot \chi_{7}+{1}\cdot \chi_{8}+{0}\cdot \chi_{9} & 9 & E(7)^{3}+E(7)^{4} & E(7)^{2}+E(7)^{5} & E(7)+E(7)^{6} & 1 & 0 & 0 & 0 & 0\\
{0}\cdot \chi_{1}+{0}\cdot \chi_{2}+{0}\cdot \chi_{3}+{0}\cdot \chi_{4}+{0}\cdot \chi_{5}+{0}\cdot \chi_{6}+{1}\cdot \chi_{7}+{0}\cdot \chi_{8}+{0}\cdot \chi_{9} & 9 & E(7)+E(7)^{6} & E(7)^{3}+E(7)^{4} & E(7)^{2}+E(7)^{5} & 1 & 0 & 0 & 0 & 0\\
 \hline
{0}\cdot \chi_{1}+{0}\cdot \chi_{2}+{1}\cdot \chi_{3}+{1}\cdot \chi_{4}+{1}\cdot \chi_{5}+{0}\cdot \chi_{6}+{0}\cdot \chi_{7}+{0}\cdot \chi_{8}+{0}\cdot \chi_{9} & 21 & 0 & 0 & 0 & -3 & 3 & -1 & 0 & 0\\
{1}\cdot \chi_{1}+{0}\cdot \chi_{2}+{1}\cdot \chi_{3}+{1}\cdot \chi_{4}+{1}\cdot \chi_{5}+{1}\cdot \chi_{6}+{0}\cdot \chi_{7}+{0}\cdot \chi_{8}+{0}\cdot \chi_{9} & 30 & 2 & 2 & 2 & -2 & 3 & 1 & 0 & 0\\
 \hline
{1}\cdot \chi_{1}+{0}\cdot \chi_{2}+{0}\cdot \chi_{3}+{0}\cdot \chi_{4}+{0}\cdot \chi_{5}+{0}\cdot \chi_{6}+{0}\cdot \chi_{7}+{0}\cdot \chi_{8}+{0}\cdot \chi_{9} & 1 & 1 & 1 & 1 & 1 & 1 & 1 & 1 & 1\\
{0}\cdot \chi_{1}+{1}\cdot \chi_{2}+{1}\cdot \chi_{3}+{1}\cdot \chi_{4}+{1}\cdot \chi_{5}+{0}\cdot \chi_{6}+{0}\cdot \chi_{7}+{0}\cdot \chi_{8}+{0}\cdot \chi_{9} & 28 & 0 & 0 & 0 & -4 & 1 & -1 & 1 & -1\\
\hline

\end{array}\)\\
\ \\
\ \\
$P_{1} = Group( [ () ] )\cong$ 1\ \\
$P_{2} = Group( [ (1,6,5)(2,4,9)(3,8,7) ] )\cong$ C3\ \\
$P_{3} = Group( [ (1,6,5)(2,4,9)(3,8,7), (1,3,9,6,8,2,5,7,4) ] )\cong$ C9\ \\
\ \\
$N_{1} = Group( [ (1,2)(3,4)(6,7)(8,9), (1,3,2)(4,5,6)(7,8,9) ] )\cong$ PSL(2,8)\ \\
$N_{2} = Group( [ (1,6,5)(2,4,9)(3,8,7), (2,8)(3,4)(5,6)(7,9), (1,2)(3,8)(4,5)(6,9) ] )\cong$ D18\ \\
$N_{3} = Group( [ (1,3,9,6,8,2,5,7,4), (1,6,5)(2,4,9)(3,8,7), (2,8)(3,4)(5,6)(7,9) ] )\cong$ D18\end{tabular}
\end{document}
