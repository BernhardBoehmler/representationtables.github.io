\documentclass[varwidth=\maxdimen,border=10]{standalone}
\begin{document}
The group $G$ is isomorphic to the group labelled by\ [ 8, 4 ]\ in the Small Groups library.\\
Ordinary character table of $G$\ $\cong$\ Q8:\\
\begin{center}
\begin{tabular}{@{}l@{}l@{}l@{}}
\hline
\(\begin{array}{|l|ccccc|}
  & 1a & 2a & 4a & 4b & 4c\\ \hline
\chi_{1} & 1 & 1 & 1 & 1 & 1\\
\chi_{2} & 1 & 1 & 1 & -1 & -1\\
\chi_{3} & 1 & 1 & -1 & 1 & -1\\
\chi_{4} & 1 & 1 & -1 & -1 & 1\\
\chi_{5} & 2 & -2 & 0 & 0 & 0\\
\hline
\end{array}\)\\
\end{tabular}
\end{center}
\begin{tabular}{@{}l@{}l@{}l@{}l@{}l@{}l@{}l@{}l@{}l@{}l@{}l@{}l@{}l@{}l@{}l@{}l@{}}
Trivial source character table of $G$\ $\cong$\ Q8 at\ $p=2$:\\
\(\begin{array}{|l|c|c|c|c|c|c|}
\hline
\textup{Normalisers}\ N_i & \multicolumn{1}{c|}{N_{1}} & \multicolumn{1}{c|}{N_{2}} & \multicolumn{1}{c|}{N_{3}} & \multicolumn{1}{c|}{N_{4}} & \multicolumn{1}{c|}{N_{5}} & \multicolumn{1}{c|}{N_{6}}\\ \hline
p\textup{-subgroups\ of\ } G\ \textup{up\ to\ conjugacy\ in\ } G & \multicolumn{1}{c|}{P_{1}} & \multicolumn{1}{c|}{P_{2}} & \multicolumn{1}{c|}{P_{3}} & \multicolumn{1}{c|}{P_{4}} & \multicolumn{1}{c|}{P_{5}} & \multicolumn{1}{c|}{P_{6}}\\ \hline
\textup{Representatives}\ n_j\ \in\ N_i & 1a & 1a & 1a & 1a & 1a & 1a\\ \hline
{1}\cdot \chi_{1}+{1}\cdot \chi_{2}+{1}\cdot \chi_{3}+{1}\cdot \chi_{4}+{2}\cdot \chi_{5} & 8 & 0 & 0 & 0 & 0 & 0\\
 \hline
{1}\cdot \chi_{1}+{1}\cdot \chi_{2}+{1}\cdot \chi_{3}+{1}\cdot \chi_{4}+{0}\cdot \chi_{5} & 4 & 4 & 0 & 0 & 0 & 0\\
 \hline
{1}\cdot \chi_{1}+{0}\cdot \chi_{2}+{1}\cdot \chi_{3}+{0}\cdot \chi_{4}+{0}\cdot \chi_{5} & 2 & 2 & 2 & 0 & 0 & 0\\
 \hline
{1}\cdot \chi_{1}+{1}\cdot \chi_{2}+{0}\cdot \chi_{3}+{0}\cdot \chi_{4}+{0}\cdot \chi_{5} & 2 & 2 & 0 & 2 & 0 & 0\\
 \hline
{1}\cdot \chi_{1}+{0}\cdot \chi_{2}+{0}\cdot \chi_{3}+{1}\cdot \chi_{4}+{0}\cdot \chi_{5} & 2 & 2 & 0 & 0 & 2 & 0\\
 \hline
{1}\cdot \chi_{1}+{0}\cdot \chi_{2}+{0}\cdot \chi_{3}+{0}\cdot \chi_{4}+{0}\cdot \chi_{5} & 1 & 1 & 1 & 1 & 1 & 1\\
\hline

\end{array}\)\\
\ \\
\ \\
$P_{1} = Group( [ () ] )\cong$ 1\ \\
$P_{2} = Group( [ (1,4)(2,6)(3,7)(5,8) ] )\cong$ C2\ \\
$P_{3} = Group( [ (1,4)(2,6)(3,7)(5,8), (1,3,4,7)(2,5,6,8) ] )\cong$ C4\ \\
$P_{4} = Group( [ (1,4)(2,6)(3,7)(5,8), (1,2,4,6)(3,8,7,5) ] )\cong$ C4\ \\
$P_{5} = Group( [ (1,4)(2,6)(3,7)(5,8), (1,8,4,5)(2,3,6,7) ] )\cong$ C4\ \\
$P_{6} = Group( [ (1,4)(2,6)(3,7)(5,8), (1,3,4,7)(2,5,6,8), (1,2,4,6)(3,8,7,5) ] )\cong$ Q8\ \\
\ \\
$N_{1} = Group( [ (1,2,4,6)(3,8,7,5), (1,3,4,7)(2,5,6,8), (1,4)(2,6)(3,7)(5,8) ] )\cong$ Q8\ \\
$N_{2} = Group( [ (1,2,4,6)(3,8,7,5), (1,3,4,7)(2,5,6,8), (1,4)(2,6)(3,7)(5,8) ] )\cong$ Q8\ \\
$N_{3} = Group( [ (1,3,4,7)(2,5,6,8), (1,4)(2,6)(3,7)(5,8), (1,2,4,6)(3,8,7,5) ] )\cong$ Q8\ \\
$N_{4} = Group( [ (1,2,4,6)(3,8,7,5), (1,4)(2,6)(3,7)(5,8), (1,3,4,7)(2,5,6,8) ] )\cong$ Q8\ \\
$N_{5} = Group( [ (1,8,4,5)(2,3,6,7), (1,4)(2,6)(3,7)(5,8), (1,2,4,6)(3,8,7,5) ] )\cong$ Q8\ \\
$N_{6} = Group( [ (1,2,4,6)(3,8,7,5), (1,3,4,7)(2,5,6,8), (1,4)(2,6)(3,7)(5,8) ] )\cong$ Q8\end{tabular}
\end{document}
